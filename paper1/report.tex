\documentclass[sigconf]{acmart}

\usepackage{hyperref}

\usepackage{endfloat}
\renewcommand{\efloatseparator}{\mbox{}} % no new page between figures

\usepackage{booktabs} % For formal tables

\settopmatter{printacmref=false} % Removes citation information below abstract
\renewcommand\footnotetextcopyrightpermission[1]{} % removes footnote with conference information in first column
\pagestyle{plain} % removes running headers

\begin{document}
\title{Big Data Analytics and Edge Computing}


\author{Arnav Arnav}
\affiliation{%
  \institution{Indiana University, Bloomington}
  \city{Bloomington} 
  \state{Indiana}
  \country{USA} 
}
\email{aarnav@iu.edu}

\begin{abstract}
With the exponential increase in the number of connected IoT devices, the data generated by these devices has grown enormously. Sending this data to a centralized server or cloud results in enormous network traffic and may lead to failures and increased latency. One solution of this problem is to do some processing on the edge devices. This is extremely helpful in providing responsive and real time analytics.
\end{abstract}

\keywords{hid201, i523, Edge Computing, Big Data Analytics}
\maketitle
\section{Introduction}
Internet of things is rapidly gaining importance and "according to the Evans Data Corporation’s Global Developer Population and Demographics Study, some 6.2 million developers are already working on IoT applications."\cite{ibm_data_streaming_analytics} With the rapid increase in the acceptance of Internet of Things (IoT) devices across various fields in the world, ranging from industrial sensors to lifestyle and sports products, and the consequent increase in the data generated by such devices, there is a pressing demand for devices and processes that can analyze this data and provide responsive analytics.\cite{ieee_iot_cloud_analytics_newsletter}. Traditionally, IoT applications follow one of the two approaches - "cloud centric approach, where the sensing devices send data to the cloud where the analytics are perfomed or device-dentric approach, where devices have some proprietery code and perform analytics in a stand-alone manner"\cite{ieee_iot_cloud_analytics_newsletter}. Networks are largely centralized with  organizations soring all data, which may not be directly beneficial for them, in their data centers, and data flowing from the edge to the cloud on each operation\cite{ibm_iot_edge}

With increase in the number of connected devices, it gets increasingly dificult to perform  all analytics on a server in a traditional manner. Thus, edge computing involves pushing a part of this computation closer to the end user of the device, or closer to the edge\cite{wiki-edge-computing}\cite{ibm_iot_edge}. This helps reduce the cost incurred in communicating large amounts of data over the network, ensures some level of availability even when the connection to the cloud is broken and reduces cost of computation and storing data on the cloud\cite{ieee_iot_cloud_analytics_newsletter}.

\section{How Edge Computing Works}
Edge computing emerged with the development of content delivery networks (CDNs) by Akami which use nodes closer to the user to prefetch and cache web content to accelerate web throughput.Edge computing extends this concept with the help of cloud infrastructure to run arbitrary task specific code at at nodes close to the edge, typically known as cloudlets. These cloudlets usually run on a virtual machine or a light weight container for ease of isolation and resource management.\cite{satyanarayananemergence}

Proximity to the edge of the network ensures various benefits. It helps to provide highly responsive applications, by using a more powerful conputing resource near the edge and minimizing end-to-end latency, which is essential in virtual reality applications which typically require a latency of less than 16ms to appear stable.\cite{rocket-real-time-video}\cite{satyanarayananemergence} Proximity also increases scalability with the help of edge analytics which uses the cloudlet to perform first level of analytics on the sensor data and only send processed data and metadata to the cloud to reduce bandwidth usage as the number of devices increases.\cite{satyanarayananemergence}.  Decentralization of data can also provide the owners of data more control over the privacy of their data, and provide ways to safely communicate this data between various entities\cite{ibm_iot_edge}

In industrial applications like aviation where a large amount of data is generated on each flight\cite{satyanarayananemergence}, analyzing this data in a centralized manner becomes impractical. In such cases fog computing is more useful which adds a "hierarchy of elements between the edge and the cloud"\cite{rt_insights_iiot}. In industrial environments, there are a lot of different systems running new as well as legacy applications which may be proprietery and integrating these applications to provide end-to-end IoT solutions is still a challenge. Linux Foundation's EdgeX platform provides a way to simplify and standardize edge computing architectures and is gaining importance as an industrial IoT solution.\cite{rt_insights_iiot}

%% ibm / openstack details
%%

\section{Some Examples}

Simmhan describes an application that was built using Apache-NiFi, a lightweight dataflow execution engine "that classifies vehicles from video streams using a Tensorflow deep neural network encapsulated within a NiFi dataflow executing across multiple Pis. This helps with local analytics of video data streams close to the camera source, but with the flexibility of using the same deployment in the Cloud too, say, when the edge is constrained."\cite{ieee_iot_cloud_analytics_newsletter}

Yang Zhao et. al proposed an occupancy and activity monitoring appicattion with doppler sensing and edge analytics. The application uses low cost motion sensing and embedded signal processing, detection and machine learning to detect activity in real time, even when multiple people are present in a room. The dvelopers provide a web portal to help ease monitoring activity from a remote location.\citep{Dolppler-usecase}

Analysing video feeds on a large scale in real time is a challenging task. Each of the videos may be very large and a large amount of bandwidth is needed to stream the video feed to a central location which is not feasible specially if the cameras are connected wirelesly. In addition to this all of the video may not be useful and most parts of it may be discarded depending on the application. Furthermore, these applications need to provide results with low latency as important decisions often need to be made based on the output in case of surveiillance applications.\cite{rocket-real-time-video} Thus compute abilities available on cameras can be utilized to provide real time video analysis, processing the video at the camera and only communicating interesting bits to the cloud.\cite{satyanarayananemergence}

A real time video processing solution is proposed in \cite{rocket-real-time-video} that focuses on traffic plannin and safety and provide high accuracy outputs and detects anomalous traffic patternd to suggest prermptive safety measures and reduce traffic accidents and deaths.
Interactive augmented reality applications must rely on object tracking, face detection, and other video analytics to obtain sppacial knowledge, and must rely on cloudlet based edge solution to provide seemless interaction for the users.\cite{rocket-real-time-video}

Scientists at MIT's Computer Science and Artificial Intelligence Lab (CSAIL) are working on self folding printed robots and their use in saving lives as an alternative to invasive surgery procedures., which would require a cloud in the proximity as those robots and sensors would generate a large amount of data that needs to be processed very fast\cite{open_stack_living_on_edge}

Verizon created a universal cloud-in-a-box solution running Linux on a generic x86 architecture, in an OpenStack container that can put compute, storage and networking resources near the edge to support their increasing number of users and power 5G in the future.\cite{open_stack_living_on_edge}\cite{open_stack_verizon}

\section{New Approaches}
%serverless platform
\cite{ieee-serverless-platform-edge}
\cite{FADES-offloading}

%FADES offloading
% ROCKET real time edge video processing

\section{AI on the Edge}
With the emergence of decentralized aplications, smart mahines that rely on machilne learning and mesh computing to provide local real time analytics are becoming a reality. MIT's Eyeris which is an accelerator for deep neural networks uses no wifi and no data transmission. With peer to peer networks gaining importance, edge computing is vital to provide low latency applications that are decentralized. \cite{ibm_iot_edge}

Since many artificial intelligenc (AI) applications need a huge amount of processing power and rewuire a large amount of data, traditional AI applications rely on cloud servers to perform their computation. This is a serious limitation in applicaitons where connectivity is not reliable and time-critical decisions are required. \cite{ai-to-edge}iEx.ec is a company that uses Etherium blockchain to createa market for computing resources, facilitating distributed mahcine learning. \cite{iExec}

In applications like flying a swarm of drones, a loss of connectivity to the cloud can be fatal and cause disruption of the operation. Thus AI coprocessor chips that can run machine leraning algotithms can offer intelligence at the edge devices. Movidius recently announced a deep learning compute stick that can add achine learning capabilities to computers and raspberry pis as a plug and play device.\cite{ai-to-edge}

Machine learning algorithms line one-shot learning that require less data are rapidly enabling edge devices to perform intelligent tasks easily.\cite{wiki-one-shot-learning}.
Gamalon, backed by Defense Advanced Research Projects Agency (DARPA), is using Bayesean Program Synthesis to reduce the amount of data required for machine learning.\cite{ai-to-edge}  
\section{Conclusion}
With the increase in the number of conected devices and the increase in the demand of real time and interactive applcations, edge computing is a necessity and many industries are rapidly moving towards edge solutions. Although industrial IoT is gaining importance but still faces challenges with the integration of legacy applications and proprietary applications with new technology, new open source solutions are gaining importance. With the emergence of decentrallized applications and the growing importance of machine learning, edge computing is required as a foundation to move towards decentralized AI applications, that provide results in near real time.  

\begin{acks}
The author would like to thank Professor Gregor von Laszewski for providing the opportunity to study the topic in detail and for providing all the tutorials and support material needed to write the paper.

The author would also like to thank other associate instructors of the class for helping promptly with queries on piazza whih helped everyone a great deal.
\end{acks}

\bibliographystyle{ACM-Reference-Format}
\bibliography{report} 

\end{document}
